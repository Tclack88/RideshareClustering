\documentclass[11pt]{article}
\usepackage{fancyhdr}
\usepackage{graphicx}
\usepackage{color}
\usepackage{hyperref}
\usepackage{amsmath}

\usepackage[hmargin=90bp,tmargin=108bp,bmargin=72bp,
            headheight=15bp,footskip=40bp]{geometry}
%%%%%%%%%%%%%%%%%%%%%%%%%%%%%%%%%%%%%%%%%%%%%%%%%%%%%%%%%%%%%%%%%%%%%%%%%%%%%%%

%
% custom definitions
%
\newcommand\thisis{Rideshare Clustering}
\newcommand\theauthor{Austin~Taylor, Trevor~Clack}

\newcommand\sfb{\sffamily\bfseries}

\newcommand\red[1]{\textcolor{red}{\sffamily\bfseries #1}}
%%%%%%%%%%%%%%%%%%%%%%%%%%%%%%%%%%%%%%%%%%%%%%%%%%%%%%%%%%%%%%%%%%%%%%%%%%%%%%%

%
% custom heading and footer
%
\fancypagestyle{firstpg}
   {
   \fancyhf{}%
   \cfoot{\sffamily\thepage}%
   \renewcommand\headrulewidth{0bp}
   }

\pagestyle{fancy}
\lhead{\sffamily \thisis}
\chead{}
\rhead{\sffamily \theauthor}

\lfoot{}
\cfoot{\sffamily\thepage}
\rfoot{}
%%%%%%%%%%%%%%%%%%%%%%%%%%%%%%%%%%%%%%%%%%%%%%%%%%%%%%%%%%%%%%%%%%%%%%%%%%%%%%%

\begin{document}
\thispagestyle{firstpg}

\noindent
{\sffamily\bfseries\huge \thisis}\\

\noindent
{\large\sffamily \theauthor}

\vspace*{20bp}

\noindent
\textbf{Software packages:}

\hfill \break
Originally, this project used a library called "adjustText" that makes one of our plots more legible. But the library shared directories with numpy and its installation on the raspberry pi
seems to alter where essential parts of numpy are stored. Consequently, the installation of 
adjustText causes python to be unable to import numpy, and it's for this reason that we 
commented out the parts of our code that rely on adjustText. We would like to note that this 
problem seems to be related to the filepaths that are followed to find imported directories. 
It is not a problem that we've experienced on other machines and troubleshooting online has 
indicated that this is the case.

\hfill \break
\noindent
We do NOT advise doing this on the raspberry pi.

\hfill \break
\noindent
However, the package can be installed  on a debian derived linux distro using "sudo apt install adjustText"

\noindent
\hfill \break
\textbf{Project premise:}

\hfill \break
	Inspired by the stress of coordinating riders and drivers for a music club.
	This club has around 100 musicians, each of whom have changing 
	availabilities. Each weekend a subset of this (20 or so) go to perform for
	a local retirement center or homeless shelter. A "gig coordinator" is 
	assigned each week to get in contact with the drivers and riders and it
	has turned out to be quite a headache. This code is intended to automate it.

\hfill \break
\noindent
The structure of our project is fairly simple, there are five files that are used by default.
\hfill \break

main.py

clustfunc.py

Masterlist.csv

going.csv

Locations.csv

\hfill \break
Additionally, we've experimented with randomly generated data in the form of DatGen.py which
generates two csv files: SampleDatMasterList.csv and SampleDatLocations.csv to test the 
algorithm with randomly generated data. We run these through the file demo.py, rather than
main.py for ease of use, in addition to constraints we've used for testing. Ultimately, we 
built demo.py for testing and showcasing purposes.

\hfill \break
For the five main files listed above, the three .csv files are simply the data that will be read
in. The three .csv files are only used to hold data in the following form:


MasterList.csv -- date/time,email,name,latitude,longitude,car status,number of seats

going.csv -- date/time,email,attendance status(yes/no),car status,number of seats

Locations.csv -- locations,latitude,longitude

\hfill \break
\noindent
(Note: car status and number of seats are vestiges of a feature we desired to have but 
ultimately did not implement. Ideally it would have made clustering even more dynamic, allowing
for group sizes to be limited by the space available to each driver but with the approach we 
made to clustering the groups would have changed format completely).

\hfill \break
Of the other two files, main.py is where the code is stored and is the file that runs the 
program. clustfunc.py contains most of the large functions being called in main.py. We used 
filename variables at the top of the file so that the administrator can quickly change them 
without digging through the code. Despite this construction, the sample data generated by 
DatGen.py cannot be run through main.py because it lacks an event list (e.g. going.csv).
\hfill \break
We built main.py with two major steps; the first step is focused on constructing dictionaries 
with the rider, driver, and location information, and the second clusters riders to a location
and driver. Once everything has been clustered, we generate two plots, each of which emphasize
different aspects of the clustering.

\hfill \break
The bulk of the work is done in clustfunc.py where our functions are stored. There 6 functions 
that we use:

\hfill \break
adjaceancy, cluster, LocGrpMatch, DriverToGrp, clusterdraw, LocMap. 

\hfill \break
Adjaceancy, is used to build an adjaceancy matrix with all the distances between riders making 
up the elements of the matrix. Adjaceancy feeds into cluster which groups riders together by 
shortest distance and it uses "complete linkage clustering" to determine the distance between 
clusters. It is important to note that complete linkage determines the distance bewteen 
clusters with the largest distance between one of its constituents to the target. 

\hfill \break
LocGrpMatch matches each cluster to a pickup location. It does so by calculating the epicenter
of the riders and matching it to the closest pickup location. Similarly, DriverToGrp matches 
the chosen locations to the nearest driver, but it also outputs a list of the drivers that 
weren't matched with a cluster.

\hfill \break
The last two functions, clusterdraw and LocMap, are used to generate our plots. Clusterdraw 
generates a colorcoded plot that shows the connections between locations and the matching 
riders and drivers. As for LocMap, a plot is generated where each location has a list of the 
people who were clustered to it.

\hfill \break
Outside of the main files we have google forms setup to acquire data from riders and drivers.
The forms can found at:

\hfill \break
\hfill \break
\hfill \break
\hfill \break
\hfill \break
MASTERLIST:

https://docs.google.com/forms/d/e/1FAIpQLSdnKRFPaNHGaOvuUKCrbap0QG8a

Rm6CC2NxCuUMI8nu0VgDjA/viewform


\hfill \break
GOING:

https://docs.google.com/forms/d/e/1FAIpQLSfFCQh1hVi4kufxqsU-8ZwCI\_sIRgPMi
bMp3JMsRj\_oeO9K3w/viewform

\hfill \break
The master list keeps track of all the people that are in the group and allowed to ride
while the event list is used to determine who should be clustered together for that event. The 
idea is the administrator will maintain the master list, as well as posting new forms for the 
event lists.

\hfill \break
Altogether the data is collected from the forms, which is then pulled into main to create 
dictionaries. The dictionaries are used to make an adjaceancy matrix which in turn is used to 
cluster the riders together. The clusters are then matched to a pickup location and a driver,
and finally the clusters are plotted better show how people were clustered.
This document was created using "pdflatex" command operating on the .tex file




\hfill \break
\textbf{Results:}

\hfill \break
The program is run simply by running 'main.py'.

\hfill \break
As you can see in the figures (final page), Clustering was successful.
The data shown was taken from inputs in a user-friendly GoogleDoc survey.
Two surveys had to be filled:

	-One MasterList which contained the majority of the information as to where
	the riders and drivers live.

	-One "Are you going this week" sort of survey.

\hfill \break
Structuring it this way makes each week very easy since the members only need
to fill out the "hard data" just once per year at most; each subsequent
submission takes less than a minute. In order to make this algorithm less "one-and-done", we've created a 
DatGen.py and demo.py function which generate random data to illustrate the
code more dynamically.





\hfill \break
\hfill \break
\hfill \break
\hfill \break
\hfill \break
\hfill \break
\hfill \break
\hfill \break
\hfill \break
\hfill \break
\hfill \break
\textbf{Credits:}

\hfill \break
There was mauch cross over but the following list reflects the majority work

\hfill \break
Austin Taylor:
\hfill \break


		-adjaceancy

		-cluster *****

		-clusterdraw

-Latex compilation

-Iterating and finalizing main.py

\hfill \break
Trevor Clack:
\hfill \break

		-LocGrpMatch

		-DriverToGrp

		-LocMap

-Front end data acquisition (google doc forms)

-Beggining of main.py (Reading into files and creating dictionaries)

-DatGen

\hfill \break
\hfill \break
\hfill \break
\hfill \break
\hfill \break
\hfill \break
%------------------------------------------------------------------------------
\begin{figure}[h]
\begin{center}
\includegraphics[width=300bp]{ClustPlot2.eps}
\vspace{-18bp}
\end{center}
\caption[]{\label{fig:Gaussian}\small
This plot concisely communicates where all drivers and riders will meet
}
\end{figure}

%-----------------------------------------------------------------------------


%------------------------------------------------------------------------------
\begin{figure}[h]
\begin{center}
\includegraphics[width=300bp]{ClustPlot1.eps}
\vspace{-18bp}
\end{center}
\caption[]{\label{fig:Gaussian}\small
First plot generated emphasizes connections between riders and drivers to their 
pickup location
}
\end{figure}

%-----------------------------------------------------------------------------


\end{document}
